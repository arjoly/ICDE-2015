\section{Decision Trees With Sparse Input Data} \label{sec:sparse-input-dt}

\subsection{Sparse matrix format}

For memory efficiency and taking advantage of sparsity we use a data structure
called compressed sparse column (csc) matrix format. It is a general format to
represent compactly sparse matrices using three arrays: a $data$ array stores
the value of each nonzero elements, an $indices$ array stores the row index
of each nonzero elements and an $indptr$ array which stores the beginning and
end of each columns in the $data$ and the $indices$ arrays.

For instance, this $3 \times 5$ matrix
\[
\begin{bmatrix}
1 & 0 & 0 & 4 & 0 \\
0 & 0 & 0 & 5 & 0 \\
0 & 0 & 0 & 0 & 0
\end{bmatrix}
\]
is represented by the following csc matrix with arrays
\begin{align*}
data &= \begin{bmatrix}1 & 4 & 5\end{bmatrix}, \\
indices &= \begin{bmatrix}0 & 0 & 1\end{bmatrix},�\\
inptr &= \begin{bmatrix}0 & 1 & 1 & 1 & 3 & 3\end{bmatrix}.
\end{align*}

The main advantages of csc matrices are to allow fast column indexing, efficient
arithmetic and matrix operations. However, row indexing is slow. Note that a
similar row-based sparse matrix called compressed sparse row format also exists
and works under similar principles.

In order to grow decision trees on sparse input matrix, we have to require a
sparse matrix format with efficient row indexing as the tree works with subset
of the samples, and also efficient column indexing as features are randomly
sampled at each node. Furthermore, we hope to speed up the overall algorithm by
taking into account the input space sparsity. Compressed sparse column matrix
already satisfies the fast column indexing requirement. We are going to show
how to efficiently exploit the data structure as to have a fast row indexing
and use the proposed approach to speed up the overall algorithm on sparse data. At the core of our proposed method is a fast sorting algorithm (a substitute for Line~\ref{alg-line:value-extract} in Algorithm~\ref{algo:find-best-split}) that works with non-zero values of a feature, sorts the positive and negative parts separately, and rearranges the sample set accordingly. 

\subsection{Nonzero Values of $\mathcal{L}_p$}

Given the sparse matrix format, the main issue is to efficiently perform the
extraction of the sample values reaching the node (the line \ref{alg-line:value-extract}
of Algorithm \ref{algo:find-best-split}). Note that this is the only operation which requires interaction
with the input matrix data. Otherwise said for a given feature $j$, one have to be able to perform the intersection between the
sample set $\mathcal{L}_p$ which have reached the node and the $m_j=indptr[j+1] -
indptr[j]$ nonzero elements of the feature $j$ as to generate a set of
possible splitting rules. If we assume that the $indices$ of the input csc matrix
array are sorted per column, then standard intersection
algorithms have the following time complexity:
\begin{enumerate}
\item in $O(|\mathcal{L}_p| \log{m_j})$ by performing $|\mathcal{L}_p| $ binary
search on the sorted $m_j$ nonzero elements;
\item in $O(|\mathcal{L}_p|\log{|\mathcal{L}_p|} + m_j\log{|\mathcal{L}_p|})$
by sorting the sample set $\mathcal{L}_p$ and performing $m_j$ binary search on
$\mathcal{L}_p$;
\item in $O(|\mathcal{L}_p|\log{|\mathcal{L}_p|} + m_j + |\mathcal{L}_p|)$ by sorting the
sample set $\mathcal{L}_p$ and retrieving the intersection by iterating over both
arrays;
\item in $O(m_j + |\mathcal{L}_p|)$ by first creating a temporary hash table from the
one array and then checking if elements of the other array are contained
in the hash table.
\end{enumerate} 

As explained below, we will be using a hybrid solution composed of a variation of approach (4) and approach (1).  In the context of decision tree induction, the intersection operation will be repeated for each sampled feature and for various sample sets $\mathcal{L}_p$.
Taking this into account, it's possible to improve approach (4). The idea is to
maintain during the tree growth a mapping, represented at Figure
\ref{fig:mapping}, between the row index, the $indices$ array, of the csc
matrix and the position of the related samples in the sample set array
$\mathcal{L}$. Since each sample only belongs to one tree branch, a subset
$\mathcal{L}_p$ of $\mathcal{L}$ can be conveniently represented by a slice
$[start, end[$ of the array $\mathcal{L}$. Thus, it's possible to check in
$O(1)$ if the $k$-th nonzero element of the csc matrix belongs to the sample set
$\mathcal{L}_p$ by checking if $mapping[indices[k]]$ is in $[start, end[$.
Maintaining the mapping for a given position $pos$ is done in $O(1)$ by
setting $mapping[\mathcal{L}[pos]]$ to $pos$. Thus we deduce that performing
the intersection between the $indices$ array and $\mathcal{L}_p$ can be
done in $O(m_j)$.

\begin{figure}
  \centering
   \def\svgwidth{0.4\textwidth}
   \graphicspath{{images}}
   \input{images/mapping.pdf_tex}
   \caption{The array $mapping$ allow to efficiently compute the intersection between the $indices$ array of the csc matrix
            and a sample set $\mathcal{L}_p$}
   \label{fig:mapping}
\end{figure}

With the application of the mapping intersection algorithm, we can speed up the
sorting operation and splitting rule evaluation of Algorithm~\ref{algo:find-best-split}
by working separately on positive and negative values. Furthermore, it's also
possible to partition a
sample set $\mathcal{L}_p$ into two partition $\mathcal{L}_r$ and
$\mathcal{L}_r$ (line~ \ref{alg-line:partition} of Algorithm~\ref{algo:tree-induction})
given a split on feature $j$ in $O(m_j)$ instead of $O(n)$. For more details of this algorithm refer to Algorithm \ref{map}. 

In practice, the number of nonzero elements $m_j$ of feature $j$ could be a
lot bigger than the size of a sample set $\mathcal{L}_p$. This is likely to
happen near the leaf nodes. Whenever the tree is fully developed, there are
only a few   samples reaching those nodes. For optimal performance, one can use
a hybrid intersection approach which combines the previously developed mapping
intersection to approach (1) based on binary search. whenever $\mathcal{L}_p \ll
m_j$, the binary approach will be faster. For more details of this algorithm refer to Algorithm \ref{bsearch}. 

The hybrid algorithm switches between \ref{bsearch} and \ref{map} by the following rule:
 \begin{equation}
 \label{switch}
  |\mathcal{L}_p| \times \log(|\mathcal{L}_p|) + |\mathcal{L}_p| \times \log(m_j) < 0.1 m_j.
 \end{equation}
Algorithm~\ref{map} is used whenever Equation~\ref{switch} is true and Algorithm~ \ref{bsearch} is used otherwise. This is Algorithm~\ref{algo:sparse-split}.

During the tree growth, one could remember which features are constant for a
subset of the samples $\mathcal{L}_p$ and a given node $t_p$. For all
descendant of node $t_p$, this will avoid the overhead of searching for a
split where none exists for those features. 

Finally note that for testing the sparse data is flattened for efficient random
memory access.

\subsection{The Optimal Split}
The optimal split in sparsely representable data is a hybrid solution of two sorting algorithms that are well-suited for sparse csc matrices. The details of this algorithm is explained below in Algorithm \ref{algo:sparse-split}. 

\begin{algorithm}\label{algo:sparse-split}
Find the best split
\begin{algorithmic}[1]
\Function{FindBestSparseSplit}{$\mathcal{L}_p$}
   \State Let $[{start}:{end}]$ be such that $\mathcal{L}_p$ = $\mathcal{L}_{[{start}:{end}]}$
    \State $S$ = $end - start$
    \State $\text{best} = -\infty$
    \For{$j \in \{0, \ldots, m-1\}$}
	    \State $m_j$ = $indptr[j + 1] - indptr[j]$
	    \If{$|\mathcal{L}_p| \log|\mathcal{L}_p| +|\mathcal{L}_p| \log(m_j) < 0.1\times m_j$}
	    \State {$F_p, F_n$ = }
	    \State \Call{sort\_bsearch}{$X$, $start$, $end$, $mapping$, $j$}
	    \Else 
	    \State {$F_p, F_n$ = }
	    \State \Call{sort\_mapping}{$X$, $start$, $end$, $mapping$, $j$}
	    \EndIf
        \For{$\theta$ in ${F}_n$}
            \State $s$ = $\left(j, \theta\right)$
             \State $\text{score} = \Delta{}I(s, \mathcal{L}_{[start:end]})$.
            \If{$\text{score} > \text{best}$}
                \State $\text{best} =\text{score}$
                \State $s^* = s$
            \EndIf
        \EndFor
            \For{$\theta$ in ${F}_p$}
            \State $s$ = $\left(j, \theta\right)$
            \State $\text{score} = \Delta{}I(s, \mathcal{L}_{[start:end]})$.
            \If{$\text{score} > \text{best}$}
                \State $\text{best} =\text{score}$
                \State $s^* = s$
            \EndIf
        \EndFor

    \EndFor
    \State \Return $s^*$
\EndFunction
\end{algorithmic}
%}
\end{algorithm}

\begin{algorithm}\label{map}
Sorts $X_{\mathcal{L}[start:end]}$ along the $jth$ feature by moving their indices in $\mathcal{L}[start:end]$ via $mapping$, and returns the sorted positive and negative values separately. 
%\textnormal{
\begin{algorithmic}[1]
\Function{sort\_mapping}{$X$, $start$, $end$, $mapping$, $j$}
    \State $F_p = [ ]  $
    \State $F_n = [ ] $
   \State $start_p$ = $end$ 
    \State $end_n$ = $start$ 
    \State {$incides$ = X.indices}
    \State {$indptr$ = X.indptr}
    \State {$data$ = X.data}
    \For{$k \in [indptr[j] \mbox{:} indptr[j+1]]$}
    	\State $index = indices[k]$
        \State $value = data[k]$
         \If{$start\,\, \leq mapping[index] \,\, < \,\, end$}
         \State $h$ = $mapping[index]$
            	\If {$value > 0$}
	
		\State $F_p$.\Call{append}{$value$}
		\State \Call{Swap}{$\mathcal{L}$,  $h$, $start_p$, $mapping$}
	         \Else
		\State $F_n$.\Call{append}{$value$}
	         \State \Call{Swap}{$\mathcal{L}$,  $h$, $end_n$, $mapping$}
	        \State $end_n \,\, += 1$
	        \EndIf
         \EndIf
     \EndFor
        \State \Call{Sort}{$F_n$, $\mathcal{L}$, $start$, $end_n$}
        \State \Call{Sort}{$F_p$, $\mathcal{L}$, $start_p$, $end$}
        \State \Return {$F_p, F_n$}
\EndFunction
\end{algorithmic}
%}
\end{algorithm}
\begin{algorithm}\label{bsearch}
Sorts $X_{\mathcal{L}[start:end]}$ along the $jth$ feature by moving their indices in $\mathcal{L}[start:end]$, via a binary search, and returns the sorted positive and negative values separately. 
%\textnormal{
\begin{algorithmic}[1]
\Function{sort\_bsearch}{$X$, $start$, $end$, $mapping$, $j$}
    \State $F_p = [ ]  $
    \State $F_n = [ ] $ 
    \State $start_p$ = $end$ 
    \State $end_n$ = $start$ 
    \State $incides$ = X.indices
    \State $indptr$ = X.indptr
    \State $data$ = X.data
    \State $indices_j = indices[indptr[j]:indptr[j+1]]$
    \State $data_j= data[indptr[j]:indptr[j+1]]$
    \State $\mathcal{L}$ = \Call{sort}{$\mathcal{L}$, start, end} 
     \For{$h \in [start:end]$}
	   \State $index$ = $\mathcal{L}[h]$
     	   \State $i$ = \Call{BinarySearch}{$index$, $indices_j$}
	   \State Returns the position of $index$ in  $indices_j$, 
	   \State and -1 if it is not found.
	   \If {$i \neq -1$ }
	   	 \If{$data_j[i]  > 0$}
		 \State $end_p \,\, -= 1$
		\State $F_p$.\Call{append}{$value$}
		\State \Call{Swap}{$\mathcal{L}$,  $h$, $start_p$, $mapping$}
	         \Else
		\State $F_n$.\Call{append}{$value$}
	         \State \Call{Swap}{$\mathcal{L}$,  $h$, $end_n$, $mapping$}
	        \State $end_n \,\, += 1$
	        \EndIf
        		\EndIf
        \EndFor
        \State \Call{Sort}{$F_n$, $\mathcal{L}$, $start$, $end_n$}
        \State \Call{Sort}{$F_p$, $\mathcal{L}$, $start_p$, $end$}
        \State \Return {$F_p, F_n$}
\EndFunction
\end{algorithmic}
%}
\begin{algorithmic}[1]
\Function{Swap}{$\mathcal{L}$, $p_1$, $p_2$, $mapping$}
\State $\mathcal{L}[pos_1]$, $\mathcal{L}[p_2]$ = $\mathcal{L}[p_2]$, $\mathcal{L}[p_1]$
\State $mapping[\mathcal{L}[p_1]] = p_1$
\State $mapping[\mathcal{L}[p_2]] = p_2$
\EndFunction
\end{algorithmic}

\begin{algorithmic}[1]
\Function{Sort}{$F$, $\mathcal{L}$, $p_1$, $p_2$, $mapping$}
\State Sort $F$ and  $\mathcal{L}_{[p_1:p_2]}$ by increasing values of $F$
 \For{$p \in [p_1:p_2]$}
	\State $mapping[\mathcal{L}[p]] = p$
 \EndFor
\EndFunction
\end{algorithmic}

%}
\end{algorithm}


